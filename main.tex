%%%%%%%%%%%%%%%%%%%%%%%%%%%%%%%%%%%%%%%%%
% fphw Assignment
% LaTeX Template
% Version 1.0 (27/04/2019)
%
% This template originates from:
% https://www.LaTeXTemplates.com
%
% Authors:
% Class by Felipe Portales-Oliva (f.portales.oliva@gmail.com) with template 
% content and modifications by Vel (vel@LaTeXTemplates.com)
%
% Template (this file) License:
% CC BY-NC-SA 3.0 (http://creativecommons.org/licenses/by-nc-sa/3.0/)
%
%%%%%%%%%%%%%%%%%%%%%%%%%%%%%%%%%%%%%%%%%

%----------------------------------------------------------------------------------------
%	PACKAGES AND OTHER DOCUMENT CONFIGURATIONS
%----------------------------------------------------------------------------------------

\documentclass[
	12pt, % Default font size, values between 10pt-12pt are allowed
	%letterpaper, % Uncomment for US letter paper size
	%spanish, % Uncomment for Spanish
]{fphw}

% Template-specific packages
\usepackage[utf8]{inputenc} % Required for inputting international characters
\usepackage[T1]{fontenc} % Output font encoding for international characters
\usepackage{mathpazo} % Use the Palatino font

\usepackage{graphicx} % Required for including images

\usepackage{booktabs} % Required for better horizontal rules in tables

\usepackage{listings} % Required for insertion of code

\usepackage{enumerate} % To modify the enumerate environment

\usepackage{listings} % For writing code

%----------------------------------------------------------------------------------------
%	ASSIGNMENT INFORMATION
%----------------------------------------------------------------------------------------

\title{Trabajo Final Programacion en Python} % Assignment title

\author{Rogelio Hashimoto Campos} % Student name

\date{Junio 17, 2021} % Due date

\institute{Universidad de Guanajuato \\ DCI} % Institute or school name

\class{HIGI (Herramientas de la informacion)} % Course or class name

\professor{Dra. Alma Xochitl Gonzalez Morales} % Professor or teacher in charge of the assignment

%----------------------------------------------------------------------------------------

\begin{document}

\maketitle % Output the assignment title, created automatically using the information in the custom commands above

%----------------------------------------------------------------------------------------
%	ASSIGNMENT CONTENT
%----------------------------------------------------------------------------------------

\subsection*{Abstract}

The task which was given to us is to use python for the data analysis of 2 subjects, first one being free subject, I chose the analysis of the sales of a bicycle shop, the other one was about the National Electoral Institue, and what we had to do was predicting how many ballouts we need for the next 2 months for the next elections.
In the case of the Bicycle shop, I calculated what was the total unity cost and follow a tendency as well as calculating which age group is more likely to buy at the shop nd what is the revenue of each group.
In the NEI we had to read 16 different data sheets and concentrate all that info, to then make our ballout numbers prediction for January and February of the year 2020.

%------------------------------------------------




\section*{Analisis del archivo de ventas sobre bicicleta}


\begin{center}
	\includegraphics[width=0.5\columnwidth]{bicicleta.jpeg} % Example image
\end{center}
%----------------------------------------------------------------------------------------

\section*{Descripcion del Codigo}

\lstset{language=Python}
\lstset{frame=lines}
\lstset{caption={Librerias usadas.}}
\lstset{label={lst:code_direct}}
\lstset{basicstyle=\footnotesize}
\begin{lstlisting}

import numpy as np
import pandas as pd
import matplotlib.pyplot as plt

%matplotlib inline


\end{lstlisting}

Esto fueron las librerias que importe para empezar el trabajo, una vez importando todo lo que necesitaba empece a programar, cheque que se vieran todos los datos, use el codigo:

\lstset{language=Python}
\lstset{frame=lines}
\lstset{caption={Extraer y leer documento .csv}}
\lstset{label={lst:code_direct}}
\lstset{basicstyle=\footnotesize}
\begin{lstlisting}

sales = pd.read_csv(
'data/sales_data.csv',
parse_dates=['Date'])

sales.head()

sales.info()

<class 'pandas.core.frame.DataFrame'>
RangeIndex: 113036 entries, 0 to 113035
Data columns (total 18 columns):
#   Column            Non-Null Count   Dtype         
---  ------            --------------   -----         
0   Date              113036 non-null  datetime64[ns]
1   Day               113036 non-null  int64         
2   Month             113036 non-null  object        
3   Year              113036 non-null  int64         
4   Customer_Age      113036 non-null  int64         
5   Age_Group         113036 non-null  object        
6   Customer_Gender   113036 non-null  object        
7   Country           113036 non-null  object        
8   State             113036 non-null  object        
9   Product_Category  113036 non-null  object        
10  Sub_Category      113036 non-null  object        
11  Product           113036 non-null  object        
12  Order_Quantity    113036 non-null  int64         
13  Unit_Cost         113036 non-null  int64         
14  Unit_Price        113036 non-null  int64         
15  Profit            113036 non-null  int64         
16  Cost              113036 non-null  int64         
17  Revenue           113036 non-null  int64         
dtypes: datetime64[ns](1), int64(9), object(8)


\end{lstlisting}
En la siguiente imagen se puede observar Que en verdad si esta leyendo los datos y los esta separando.

\begin{center}
	\includegraphics[width=0.5\columnwidth]{Tabla1.png} % Example image
\end{center}

En este proyecto hice analisis y visualizacion de la tabla de Unidad de costo.

\lstset{language=Python}
\lstset{frame=lines}
\lstset{caption={Librerias usadas.}}
\lstset{label={lst:code_direct}}
\lstset{basicstyle=\footnotesize}
\begin{lstlisting}
In[8]: sales['Unit_Cost'].describe()

Out[8]:

count    113036.000000
mean        267.296366
std         549.835483
min           1.000000
25%           2.000000
50%           9.000000
75%          42.000000
max        2171.000000
Name: Unit_Cost, dtype: float64

In [9]:

sales['Unit_Cost'].mean()

Out[9]:

267.296365759581

In [10]:

sales['Unit_Cost'].median()

Out[10]:

9.0




\end{lstlisting}

Obtuve la infromacion de la columna unidad de costo, nos da el promedio que es 267.2963, tambien nos arroja desviacion estanda, mediana y maximo.

En las siguientes graficas nos mestra la interaccion entre la densidad que se tiene del producto y su unidad de costo, en otra grafica se tiene lo mismo pero se mustra la linea del promedio y de la mediana.
\begin{center}
	\includegraphics[width=0.5\columnwidth]{Grafica 1.png} % Example image
\end{center}

\begin{center}
	\includegraphics[width=0.5\columnwidth]{Grafica 2.png} % Example image
\end{center}

\begin{center}
	\includegraphics[width=0.5\columnwidth]{Grafica 3.png} % Example image
\end{center}

A continuacion se hizo el analisis con el grupo de edades, sobre el consumo que tienen en la tienda.

\lstset{language=Python}
\lstset{frame=lines}
\lstset{caption={Librerias usadas.}}
\lstset{label={lst:code_direct}}
\lstset{basicstyle=\footnotesize}
\begin{lstlisting}


In [15]:

sales['Age_Group'].value_counts()

Out[15]:

Adults (35-64)          55824
Young Adults (25-34)    38654
Youth (<25)             17828
Seniors (64+)             730
Name: Age_Group, dtype: int64

In [17]:

sales['Age_Group'].value_counts().plot(kind='pie', figsize=(6,6))

Out[17]:

<matplotlib.axes._subplots.AxesSubplot at 0x7f8ade8bc2e0>



\end{lstlisting}

Este codigo nos arrojo 2 graficas:

\begin{center}
	\includegraphics[width=0.5\columnwidth]{Grafica 5.png} % Example image
\end{center}

\begin{center}
	\includegraphics[width=0.5\columnwidth]{Grafica 6.png} % Example image
\end{center}

En las siguientes Lineas de codigo, jugue con la informacion y queria ver la correlacion entre los datos.

\begin{center}
	\includegraphics[width=0.5\columnwidth]{Code1.png} % Example image
\end{center}


\lstset{language=Python}
\lstset{frame=lines}
\lstset{caption={Grafica de correlacion de todos los datos.}}
\lstset{label={lst:code_direct}}
\lstset{basicstyle=\footnotesize}
\begin{lstlisting}
fig = plt.figure(figsize=(8,8))
plt.matshow(corr, cmap='RdBu', fignum=fig.number)
plt.xticks(range(len(corr.columns)), corr.columns, rotation='vertical');
plt.yticks(range(len(corr.columns)), corr.columns);
\end{lstlisting}

\begin{center}
	\includegraphics[width=0.5\columnwidth]{Grafica 7.png} % Example image
\end{center}

Por ultimo saque las ganancias que se tienen por cada grupo de edad.

\lstset{language=Python}
\lstset{frame=lines}
\lstset{caption={Grafica de correlacion de todos los datos.}}
\lstset{label={lst:code_direct}}
\lstset{basicstyle=\footnotesize}
\begin{lstlisting}

Get the mean revenue of the Adults (35-64) sales group
In [39]:

sales.loc[sales['Age_Group'] == 'Adults (35-64)', 'Revenue'].mean()

Out[39]:

762.8287654055604

How many records belong to Age Group Youth (<25) or Adults (35-64)?
In [43]:

sales.loc[(sales['Age_Group'] == 'Youth (<25)') | (sales['Age_Group'] == 'Adults (35-64)')].shape[0]

Out[43]:

73652

Get the mean revenue of the sales group Adults (35-64) in United States
In [44]:

sales.loc[(sales['Age_Group'] == 'Adults (35-64)') & (sales['Country'] == 'United States'), 'Revenue'].mean()

Out[44]:

726.7260473588342

Increase the revenue by 10% to every sale made in France
In [45]:

sales.loc[sales['Country'] == 'France', 'Revenue'].head()

Out[45]:

50     787
51     787
52    2957
53    2851
60     626
Name: Revenue, dtype: int64

In [46]:

#sales.loc[sales['Country'] == 'France', 'Revenue'] = sales.loc[sales['Country'] == 'France', 'Revenue'] * 1.1

sales.loc[sales['Country'] == 'France', 'Revenue'] *= 1.1

In [47]:

sales.loc[sales['Country'] == 'France', 'Revenue'].head()

Out[47]:

50     865.7
51     865.7
52    3252.7
53    3136.1
60     688.6
Name: Revenue, dtype: float64


\end{lstlisting}


%------------------------------------------------

\section*{Analisis del archivo de INE y prediccion de Casillas.}

Se usaron las siguientes librerias:

\lstset{language=Python}
\lstset{frame=lines}
\lstset{caption={Grafica de correlacion de todos los datos.}}
\lstset{label={lst:code_direct}}
\lstset{basicstyle=\footnotesize}
\begin{lstlisting}
##Importar librerias a utilizar 
import numpy as np
import matplotlib.pyplot as plt
#from scipy.optimize import minimize
import os
import glob 
import pandas as pd
import re
import math
\end{lstlisting}

Y se usaron las siguientes lineas de codigo para poder leer los documentos .csv:

\lstset{language=Python}
\lstset{frame=lines}
\lstset{caption={Grafica de correlacion de todos los datos.}}
\lstset{label={lst:code_direct}}
\lstset{basicstyle=\footnotesize}
\begin{lstlisting}
import glob
files=glob.glob("./higi/*.txt")

date=[]
date_=[]
files_=[]

for i,file in enumerate(files):
date.append(re.findall(r'\d+',file)[0])


temp=sorted(range(len(date)), key=date.__getitem__)

for i in temp:
date_.append(date[i])
print(date[i],files[i])
files_.append(files[i])
\end{lstlisting}

Se uso el codigo para poder acomodar los archivos por meses y asi extraerlos en orden.
Se obtuvo la lista nominal y la lista nacional de cada seccion de Guanajuato, y se grafico.

\lstset{language=Python}
\lstset{frame=lines}
\lstset{caption={Grafica de correlacion de todos los datos.}}
\lstset{label={lst:code_direct}}
\lstset{basicstyle=\footnotesize}
\begin{lstlisting}
for i,file in enumerate(files_):
data=pd.read_csv(file)
data=data[1:]
data=data[data['ENTIDAD']==11][1:]
mpo=data.groupby(['MUNICIPIO']).sum()
if i==0 :
if 'LISTA_NAL' in mpo.columns:
df_mpo = pd.DataFrame(mpo['LISTA_NAL'])
if 'LISTA_NACIONAL' in mpo.columns:
df_mpo = pd.DataFrame(mpo['LISTA_NACIONAL'])
if 'LISTA' in mpo.columns:
df_mpo = pd.DataFrame(mpo['LISTA'])
else:
if 'LISTA_NAL' in mpo.columns:
df_mpo[date_[i]]=mpo['LISTA_NAL']
if 'LISTA_NACIONAL' in mpo.columns:
df_mpo[date_[i]]=mpo['LISTA_NACIONAL']
if 'LISTA' in mpo.columns:
df_mpo[date_[i]]=mpo['LISTA']ing}

plt.figure(figsize=(6,6))

for i in range(46):
plt.plot(df_mpo.iloc[i])
plt.yscale('log')
\end{lstlisting}

\begin{center}
	\includegraphics[width=0.5\columnwidth]{Tabla 2.png} % Example image
\end{center}




\end{document}
